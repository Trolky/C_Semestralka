% Specifikace třídy dokumentu a základní velikosti písma.
\documentclass[12pt, a4paper]{report}

% Podpora češtiny
\usepackage[utf8]{inputenc}
\usepackage[IL2]{fontenc}
\usepackage[czech]{babel}

% Okraje stránky
\usepackage[
    left=30mm, 
    right=30mm, 
    top=40mm, 
    bottom=30mm,
%    twoside, % Při oboustranné sazbě si zkuste nastavit right=25mm a left=35mm.
    % showframe % Vykreslí okraje stránky.
]{geometry}

% Americký styl odstavců (mně se tento styl líbí o poznání více)
\usepackage{parskip}

% Sazba obrázků
\usepackage{graphicx}
\graphicspath{{Images/}} % Při vkládání obrázků se bude prefixovat tato relativní cesta.

% Barvičky a barevný text
\usepackage[usenames, dvipsnames]{xcolor}
\definecolor{success}{RGB}{25, 135, 84}

% Velice praktický balík pro sazbu matematiky.
\usepackage{amsmath}

% Možnost změny rozvržení kapitol (pokročilejší záležitosti).
% Význam argumentů je popsán v dokumentaci balíku `titlesec`: mirrors.ctan.org/macros/latex/contrib/titlesec/titlesec.pdf
%\usepackage[explicit,compact]{titlesec}
%\titleformat{\chapter}[block]{\bfseries\huge}{\thechapter}{3ex}{#1}
%\titlespacing{\chapter}{0mm}{\baselineskip}{\baselineskip}

% Při použití tohoto balíku začnou fungovat odkazy v textu.
% Zkuste třeba kliknout na odkazy v textu (např. "1.1" na straně 2) nebo v seznamu obrázků/tabulek.
% `hidelinks` skryje ošklivé výchozí rámečky kolem odkazů.
\usepackage[hidelinks]{hyperref}

% Zařadí literaturu a seznamy do Obsahu.
\usepackage[nottoc]{tocbibind}

% Ukázka tvorby uživatelského makra
\newcommand{\lawyertalk}{\tiny}

% Začátek dokumentu
\begin{document}

% Titulní strana (prostředí minimálně odstraní číslo strany)
\begin{titlepage}
    \centering      % Odtud do konce prostředí bude vše na středu,
    \Large          % velkými písmeny
    \sffamily       % a bezpatkovým písmem.

    %Vložení obrázku (ze složky `Images`)
    \includegraphics[width=.7\textwidth]{fav}

    Semestrální práce z předmětu

    % Prázdná mezera mezi řádky znamená nový odstavec.
    Programování v jazyce ANSI C
    
    % Vertikální mezera 18 mm.
    \vspace{18mm}
    {\Huge\bfseries Identifikace spamu naivním bayesovským klasifikátorem}

    \vspace{18mm}
    \today                          % Čas je získán ze systému.

    \vfill                          % Vyplní prostor
    \raggedright                    % Vše bude zarováno do leva.
    \textsl{Autor:}\\   % Vtípek z přednášky + ukázka tvorby makra a přidání sémantiky do stylu textu.
    Martin Reich\\               % Příkaz \\ provede násilný zlom řádky.
    A22B0123P\\
    \texttt{reichm@students.zcu.cz}
    
    \vspace{\baselineskip}
    \textsl{Vyučující:}\\
    Ing. Kamil Ekštein, Ph.D.\\
    \texttt{kekstein@kiv.zcu.cz}
\end{titlepage}

% Takhle snadno se vytvoří obsah dokumentu.
% V případně chybějícího záznamu zkuste dokument přeložit vícekrát.
\tableofcontents

% Ukázka odstranění čísla stránky -- první stránka obsahu ale má být číslovaná!
%\thispagestyle{empty}

% Sazba nové kapitoly (ve vašem případě zde bude zřejmě zkrácená verze "Zadání")
\chapter{Úvod}
Příliš žluťoučký kůň úpěl ďábelské ódy.

\section{Sekce o něčem}
\label{sec:necem}
{\color{success}\bfseries Feministé odpustí...}

Lorem ipsum dolor sit amet, consectetuer adipiscing elit. Cum sociis natoque penatibus et magnis dis parturient montes,
nascetur ridiculus mus. Aliquam ornare wisi eu metus. Maecenas lorem. Fusce aliquam vestibulum ipsum. Suspendisse nisl.
Nulla non lectus sed nisl molestie malesuada. Integer malesuada. Itaque earum rerum hic tenetur a sapiente delectus, ut
aut reiciendis voluptatibus maiores alias consequatur aut perferendis doloribus asperiores repellat. Fusce wisi. Mauris
metus. Integer vulputate sem a nibh rutrum consequat. Maecenas lorem. In convallis. Tutos is visiblos inos obrazos
\ref{fig:fav} v~sekocos \ref{sec:necem} nasos straniaros \pageref{fig:fav}.

Suspendisse sagittis ultrices augue. Duis risus. Nam quis nulla. Vivamus luctus egestas leo\footnote{Poznámka pod čarou}.
Praesent id justo in neque elementum ultrices. Morbi imperdiet, mauris ac auctor dictum, nisl ligula egestas nulla, et
sollicitudin sem purus in lacus. Tabulus is viz \ref{tab:stehno}. Citátiére nikdymente nebylonáre jednoduššůre
\cite{KnuthAOCP2}.

Cum sociis natoque penatibus et magnis dis parturient montes, nascetur ridiculus mus. Aliquam ornare wisi eu metus.
Maecenas lorem. Fusce aliquam vestibulum ipsum \cite{KnuthAOCP2}.

\begin{figure}
    \centering
    \includegraphics[width=.3\textwidth]{fav}
    \caption{Logo naší milované fakulty}
    \label{fig:fav}
\end{figure}

% Doporučuji tento generátor: https://www.tablesgenerator.com/
% Generátor sázet tabulky i pomocí balíku `booktabs`, které vypadají opravdu lépe.
\begin{table}%[h]
    \centering
    \caption{Tabulka jako býk...}
    \begin{tabular}{ccp{3cm}}
        \hline
        \bfseries Sloupec 1 & \itshape Sloupec 2 & Sloupec 3 \\ \hline\hline
        1 & 2 & 3 \\ 
        4 & 5 & 6 \\
        7 & 8 & 9 \\ \hline
    \end{tabular}
    \label{tab:stehno}
\end{table}

\subsection{Podsekce o něčem}

\subsubsection{Podpodsekce o něčem}

\section{Sekce o něčem dalším}

\section{Sekce o matice}
Suspendisse sagittis ultrices augue. Duis risus. Nam quis nulla. Vivamus luctus egestas leo. Praesent id justo in neque
elementum $$\sin(a_{1}^2) = \sum_{0}^{2\pi}\frac{1}{n^2}$$ ultrices. Morbi imperdiet $\frac{1}{n^2}$, mauris ac auctor
dictum, nisl ligula egestas nulla, et sollicitudin sem purus in lacus \ref{equ:vzorec}.
\begin{equation}
    \gamma = c_{NB} = \arg\max_{c_i \in C}\left( P(c_i) \times \sum_{k\,\in\,\text{\bfseries pozice}}
        \log (P(\langle\text{word}_k | c_i \rangle))\right)
    \label{equ:vzorec}
\end{equation}

% Prostředí `equation*` odebírá label (číslo rovnice).
\begin{equation*}
    \iiint\limits_{V} \frac{1}{x^2}\,\mathrm{d}x
\end{equation*}

\section{Seznamy}
Až přijdeš domů tak číslovaně:
\begin{itemize}
    \item zatop,
    \item uvař,
    \item umej nádobí
    \item a jdi spát!
\end{itemize}

Až přijdeš domů tak číslovaně:
\begin{enumerate}
    \item zatop,
    \item uvař,
    \item umej nádobí
    \item a jdi spát! 
\end{enumerate}

\section{Šílenosti}
\v{m} \'{x}

\chapter{Závěr}
Deklarativní popis dokumentu v obyčejném "texťáku" (který může být mj. generovaný strojem -- třeba Doxygenem),
automatické dělení slov, podpora vektorové grafiky, odkazování v~textu, generování obsahu, seznamů obrázků a~tabulek,
geniálně prostá a~neprůstřelná bibliografie... nic z toho Vám jiné DTP softwary v dostatečné kvalitě nikdy nenabídnou.

% První způsob sazby bibliografie.
% https://www.overleaf.com/learn/latex/Bibliography_management_in_LaTeX
%\begin{thebibliography}{1}
%    \bibitem{KnuthAOCP2} D. E. Knuth. \textit{The Art of Computer Programming, Volume 2 (3rd Ed.): Seminumerical
%    Algorithms.} Addison-Wesley Longman Publishing Co., Inc., Boston, MA, USA, 1997.
%\end{thebibliography}

% Druhý způsob sazby bibliografie, kde se už nemusím starat o řezy písma, formát citace apod.
% Strunkturovaný soubor `literatura.bib` navíc můžu snadno použít i v jiných dokumentech.
% Navíc vám chybějící údaje kompilátor omlátí o hlavu, a to se vždy vyplatí!
\bibliographystyle{unsrt}
\bibliography{literatura}

% Seznamy obrázků a tabulek
\listoffigures
\listoftables

% Začátek příloh dokumentu
\appendix
\chapter{První kapitola v příloze dokumentu}

\end{document}
